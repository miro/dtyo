
 \section{Testing}

Testing is one of the oldest forms of software defect removal. It has been the most important category of defect removal since the beginning of the software industry, and in many cases even today, it is the only defect removal activity used. Several aspects of testing is covered widely in literature such as testing itself, test case design, test libraries and others. There are also a variety of standards and certifications offered by several companies and groups. Considering the penetration and importance of testing, there is surprisingly low amounts of quantitative data available on testing and test results. Quantitative data in this context means information about numbers of test cases used, numbers of defects found and other information that can be presented in numbers. In addition to the amount of data, the variety of business sizes is not as wide as it could be. The reason for this is that small companies rarely evaluate or benchmark let alone document the results with sufficient precision.

 
% Definition??

% Quantitive data?? Onko olennaista?


% -Black box / Glass box
% Functional / Nonfunctional
% Automated / Manual
% General / Automatic / Specialized / User
% 
% Testing by developers vs test personnel p.342




 \begin{itemize}
 
 \item Crash, Smoke and Kattava testaus
 
 \item ECO: Chapter 5

 \item ROI: Three main activities: Review, process audit and testing
 
 \end{itemize}

% Test stage frequency p.289
% 
% Average test stages: Subroutine, unit, function, regression, system, beta test
% Defect removal efficiency for these 6 usually 75%-85%. < 1000fp sometimes >90%
% Truly universal: Subroutine and unit tests (+system test with different names)
% 
% p.291 function points vs test stages
% 

% Relevant testing stages for small applications:
% Subroutine testing
% Unit testing?
% New function testing
% Regression testing


 \subsection{Subroutine Testing}

 \subsection{Unit Testing}

% TDD

 \subsection{New Function Testing}

 \subsection{Regression Testing}

 \subsection{Integration Testing}

 \subsection{System Testing}

 \subsection{Agile Testing}