\section{Post Release}

After a software product has been released to the market, it practically always still has defects in it. IBM calls these defects latent defects, because before the release, these defects have not yet been found as problems to customers. The existence of these defects is due to the imperfect effectiveness of the defect removal. Usually the defect removal efficiency is around 85\% and virtually never reaches 100\%.

\subsection{Latent defects}

Some latent defects can be defects found during the development or testing but ones that have not been repaired before the release of the software. Other defects were present in the application, but not discovered by the developers or test personnel. Furthermore, some defects can be originated from new development or other defect repairs in the form of bad fixes. The last two weeks before the release can bring in from about 1\% to even 5\% of delivered defects.

In a traditional development of commercial software, most of the latent defects found in after the release were those that hadn't been found and removed in the development and testing phase. In the more recent history, some vendors have started to release software with remarkable amounts of known, but not removed, defects. In small applications, below 1000 function points, there might be a handful of latent defects present. In larger systems, hundreds of latent defects can be released with the software. Moreover, in massive applications, like Windows 7 or SAP, the amount of known latent defects can sometimes be counted in thousands.

The motivation behind releasing a software with known latent defects appears to be compiled from three factors: first, the aspiration for achieving earlier release dates. Second, an assumption that a quick subsequent release will fix the defects. Third, the utilization of the skills of the customers for finding and repairing defects. The latest of the three can include a strategic offer for customers to get a compensation for repairing or identifying flaws.

This trend of releasing a software knowingly with defects has made customers skeptical about buying or installing the first release of a new software. Some customers prefer to wait for a second release, assuming the latter versions have many of the latent defects removed.
identifying security flaws


\subsection{Defect severity levels}

Because of the potential high amount of defects combined with limited amount of resources for removing them, some system for categorizing the defects on the basis of seriousness is needed. One of the oldest methods for assigning severity levels to defects is the IBM severity scale, which dates back to 1950s. It is still probably the most used severity scale.

The IBM severity scale contains four levels of severities and four other categories of defects. The defects in the first severity level cause that the software does not operate at all. Level 2 defects are disruptions or errors in major features. Level 3 contains minor disruptions, with which the software is still usable. Severity level 4 defects cause cosmetic errors that does not impact the operation of the software.

The other categories in the IBM severity scale consist of four levels existing for convenience. Invalid defect level contains problems that are caused by hardware or other software. Duplicate defect is a category for additional reports of a known defect. Abeyant defect contains defects that cannot be reproduced. Improvement category is for reported defects which are actually suggested improvements.

The usage of the severity scale is for arranging the defects to an order in which they are repaired. Defects in the higher levels are more important to customers than the low-severity defects. Because of this, the group responsible for removing post-release defects use more effort to the higher level defects. The defects in the highest levels may even require temporary fixes to allow the continued usage of the software.

% Structural quality !?!

\subsection{Maintainability}

%Maintainability
%	Maintenance assignment scope
%	Cyclomatic complexity
%	Entropy or rate of structural decay
%
%Positive impact on maintainability
%	Training
%	Structural diagrams
%	Comments clarity
%	No error-prone modules
%	Maintenance tools
%	Maintenance workloads
%	Programming languages
%


\subsection{First year discovery rates}

\subsection{Fixers: Development personnel or Maintenance specialists}

\subsection{Costs of Post-Release Defects in small application}

