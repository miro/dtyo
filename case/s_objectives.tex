
 \section{Project objectives}

The project was built on an idea from MukavaIT. The big picture and the goals of the project were specified in advance and introduced to the developing team. Lower level specifications and implementation details were designed in cooperation between MukavaIT and the development team.

The objective of the project was to create a new kind of recording system for kindergartens. The system would allow the tracking of the arrival, departure and absences of the children in the groups. The main goal was to make a user-friendly system which connects parents, employees of the kindergartens and the administrative personnel in the municipality. The connection between the stakeholders would enable the planning of the need for day care and the real-time tracking of the nurses and children. The eventual goal of the system would be the dynamic hour-based billing models. In addition to this, the system would produce several types of reports for the administrative personnel for optimizing the daycare system.

The most important functionality of the system would be that the nurses are able to record the presences of the children with mobile devices. The recordings will be collected to a database serving different services for nurses, parents and administrative personnel in the kindergartens. The services would include a mobile client for nurses, a desktop client for parents and another desktop client for administrative personnel. A common goal in which these clients are aimed at, is to bring the daycare system towards a genuinely transparent process, which encourages the parents to participate more in the early childhood education.

During the development, a better understanding of the market was achieved. It was learned that reaching towards the hour-based billing models and the report generation was not as necessary as it first seemed. The stakeholders became interested already as the first version of the system was released into pilot use. The features that allowed the parents to communicate with the nurses were something the parents were excited of. In addition, the recording and checking the presences, communicating with the parents and storing the information about the children all in the same system were inspirational for the nurses.

This understanding of the market was not present when deciding the priorities of the upcoming features, so some of the effort put to reporting and other features could have been allocated to more appealing features.

 \section{Team and Phases VAI Project Execution}

The project was divided into multiple phases. In the scope of this thesis, five phases are observed. Each of the phases had predefined objectives and a time frame. The project was executed using agile practices, so the phases were divided in sprints lasting usually a week. A weekly session, with both the development team and the customers present, was arranged in where the previous sprint was reviewed and the next sprint was planned. The contents of the sprints and the backlog were managed with Pivotal Tracker, a tool for agile project management.

The development team consisted of a few developers, a user experience and design specialist and a project manager. The size and structure of the development team stayed rather consistent, but the individuals belonging to the team were changed several times during the project. Also the pace of the development varied somewhat between the phases.

In the description of each of the phases, the amount of functionality is approximated by using a conversion from the amount of source code lines to function points. The user interface is implemented with web technologies, so the amount of Javascript code is used. For the backend, the implementation language is Groovy, which is assumed to be close to Java in function points. \textbf{TODO: viite http://www.qsm.com/resources/function-point-languages-table}

% Content
\textbf{Phase one.} The feature content of the first phase included the implementation of the mobile client and administration interface. The mobile client was for recording the presences of the children and the nurses. The administration interface included the principal editing of kindergartens, care groups, children and nurses.

% Time frame, team, budget, new features and bugs, function points
The phase lasted 87 calendar days and the team in the first phase consisted of three developers, a UX specialist and a project manager. Budget of the phase was 26\% of the total budget of the five stages presented here. The phase included 86 new features registered in the Pivotal Tracker. 17 bugs were found during this phase. New source code were introduced worth of 130 function points.

% Things affecting quality
In the first phase, the project quality was affected by the introduction of the continuous integration environment and a user interface testing framework. 
The continuous integration was handled with a product called Bamboo. It was configured to track the changes in the repositories of both the mobile client and the desktop client. Whenever a change was pushed into the repository, the continuous integration system would run the tests and update the testing environment. In addition, Bamboo could update the production server by running the job manually. Testing of the mobile client was started with a small set of tests running with CasperJS.

\textbf{!!!!! Spock testing framework?? !!!!}

% Content
\textbf{Phases two and three.} These two phases shared some goals so they are merged in the project records. During these phases, the development targets were in the communication, planning and administration. Communication between the kindergarten and children's home was implemented to the mobile client and the desktop client for the parents. For the parents, a tool for planning the future presences of the children was implemented. These presences were intended to help the personnel of the kindergarten plan the future shifts of the nurses. The last big feature for these phases was a desktop client for the administrative personnel of the municipality. This manager client for the kindergarten management included features for browsing and editing the kindergarten information, including the information about the children and the nurses. In addition, this manager client could generate reports describing the recent history of the kindergarten.

% Time frame, team, budget, new features and bugs, function points
Phases two and three took 105 calendar days in total. At the beginning of the phase two, the team consisted of two developers, a UX specialist and a project manager. After a one third of the period, a third developer was included in the team. These two phases took approximately 56\% of the total budget. According to the Pivotal Tracker, 55 new features were implemented and 2 bugs were fixed. New source code was introduced worth of 285 function points, totaling 415 function points for the project so far.

% Things affecting quality
During these phases, the quality of the project shifted in different directions. A staging server environment identical to the production server was created. The staging server was integrated with the continuous integration server. The staging server had two purposes: the customer could demonstrate the application to potential customers and some system testing could be done in the staging server.

In addition to the introduction of the staging server, the quality of the project was affected by problems with testing. Testing the user interfaces of the application became an issue, which eventually lead to total disabling of the user interface tests. The issues were related to gradually degrading tests, which in the end impeded the usage of the continuous integration environment.

% Content
\textbf{Phase four.} In the fourth phase the main goal was to implement the support for the features and APIs for family daycare. These included several exceptions to the ordinary kindergarten operation, for example calculating the compensation of expenses for the care provider.

% Time frame, team, budget, new features and bugs, function points
The time frame for the fourth phase was 40 calendar days. The team included two full-time developers, one part-time developer, a UX specialist available when needed and a project manager. The budget of the fourth phase took only 6\% of the total budget. The history data from Pivotal Tracker reveals 8 features and 3 bugs for this phase. The amount of source code raised by 50 function points to a total of 475 function points.

% Things affecting quality
No major changes to the quality aspects of the project in the fourth phase. User interface tests were kept disabled and testing of the new features was handled manually.

% * Vaihe 5
% 	* Aikataulu: 15.7.2013 - 16.7.2013
% 	* Budjetti: 12%   30 000€
%	* Function points 				500 (+25)
% 	* Pääkontentti: 
% 		* Läsnäolotietomallin uudistus, läsnäolotietojen esitys ja muokkaus, kulukorvaukset kunta UI:hin 
% 	* Tiimi
% 		* Osmo (osa-aikainen)
% 		* Mike
% 		* Miro
% 		* Kimmo
% 		* (Elice)

% Content
\textbf{Phase five.} In the fifth phase, REFAKTOROINTI?????? TODO-HÄLÄRM!!


% Time frame, team, budget, new features and bugs, function points
The fifth phase lasted 32 calendar days. The budget for this phase was around 12\% of the total budget. Three developers, an on-demand UX specialist and a part-time project manager aggregated the team for this phase. Pivotal Tracker contains only one feature and one bug allocated for this phase. The veracity of these numbers can be questioned and it is possible that the usage of the tracker was neglected for this phase. \textbf{TODO!?!} A total of 25 function points were implemented as new source code and after this phase the total function points reached an even 500 function points. 

% Things affecting quality
Some major refactoring tasks of the application were done in this phase, so the structure of the application was improved. Thus the overall quality of the application should have became higher.

