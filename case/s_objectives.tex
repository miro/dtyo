
 \section{Project objectives}

The project was built on an idea from MukavaIT. The big picture and the goals of the project were specified in advance and introduced to the developing team. Lower level specifications and implementation details were designed in cooperation between MukavaIT and the development team.

The objective of the project was to create a new kind of recording system for kindergartens. The system would allow the tracking of the arrival, departure and absences of the children in the groups. The main goal was to make a user-friendly system which connects parents, employees of the kindergartens and the administrative personnel in the municipality. The connection between the stakeholders would enable the planning of the need for day care and the real-time tracking of the nurses and children. The eventual goal of the system would be the dynamic hour-based billing models. In addition to this, the system would produce several types of reports for the administrative personnel for optimizing the daycare system.

The most important functionality of the system would be that the nurses are able to record the presences of the children with mobile devices. The recordings will be collected to a database serving different services for nurses, parents and administrative personnel in the kindergartens. The services would include a mobile client for nurses, a desktop client for parents and another desktop client for administrative personnel. A common goal in which these clients are aimed at, is to bring the daycare system towards a genuinely transparent process, which encourages the parents to participate more in the early childhood education.

During the development, a better understanding of the market was achieved. It was learned that reaching towards the hour-based billing models and the report generation was not as necessary as it first seemed. The stakeholders became interested already as the first version of the system was released into pilot use. The features that allowed the parents to communicate with the nurses were something the parents were excited of. In addition, the recording and checking the presences, communicating with the parents and storing the information about the children all in the same system were inspirational for the nurses.

This understanding of the market was not present when deciding the priorities of the upcoming features, so some of the effort put to reporting and other features could have been allocated to more appealing features.

 \section{Team and Phases VAI Project Execution}

The project was divided into multiple phases. In the scope of this thesis, five phases are observed. Each of the phases had predefined objectives and a time frame. The project was executed using agile practices, so the phases were divided in sprints lasting usually a week. A weekly session, with both the development team and the customers present, was arranged in where the previous sprint was reviewed and the next sprint was planned. The contents of the sprints and the backlog were managed with Pivotal Tracker, a tool for agile project management.

The development team consisted of a few developers, a user experience and design specialist and a project manager. The size and structure of the development team stayed rather consistent, but the individuals belonging to the team were changed several times during the project. Also the pace of the development varied somewhat between the phases.

% Content
\textbf{Phase one.} The feature content of the first phase included the implementation of the mobile client and administration interface. The mobile client was for recording the presences of the children and the nurses. The administration interface included the principal editing of kindergartens, care groups, children and nurses.

% Time frame, team, budget, new features and bugs, function points
The phase lasted 87 calendar days and the team in the first phase consisted of three developers, a UX specialist and a project manager. Budget of the phase was 26\% of the total budget of the five stages presented here. The phase included 86 new features registered in the Pivotal Tracker. 17 bugs were found during this phase. New source code were introduced worth of 130 function points.

% Things affecting quality
In the first phase, the project quality was affected by the introduction of the continuous integration environment and a user interface testing framework. 
The continuous integration was handled with a product called Bamboo. It was configured to track the changes in the repositories of both the mobile client and the desktop client. Whenever a change was pushed into the repository, the continuous integration system would run the tests and update the testing environment. In addition, Bamboo could update the production server by running the job manually. Testing of the mobile client was started with a small set of tests running with CasperJS.

\textbf{!!!!! Spock testing framework?? !!!!}

% * Vaihe 2-3
% 	* Aikataulu: 19.2.2013-4.6.2013
% 	* Budjetti: 56%   144 840€ (136 000€)
%	* Function points 				415 (+285)
% 	* Pääkontentti: 
% 		* Kommunikointi päiväkodin ja kodin välillä
% 		* Hoitosuunnitelmat
% 		* Päiväkodin käyttöliittymä tietojen selaamiseen ja muokkaukseen sekä raportointiin 
% 	* Tiimi
% 		* Osmo
% 		* Mike
% 		* Kalle
% 		* Mikko (huhtikuu ->)
% 		* Elice

% Content
\textbf{Phases two and three.} These two phases shared some goals so they are merged in the project records. During these phases, the development targets were in the communication, planning and administration. Communication between the kindergarten and children's home was implemented to the mobile client and the desktop client for the parents. For the parents, a tool for planning the future presences of the children was implemented. These presences were intended to help the personnel of the kindergarten plan the future shifts of the nurses. The last big feature for these phases was a desktop client for the administrative personnel of the municipality. This manager client for the kindergarten management included features for browsing and editing the kindergarten information, including the information about the children and the nurses. In addition, this manager client could generate reports describing the recent history of the kindergarten.

% Time frame, team, budget, new features and bugs, function points
Phases two and three took 105 days total. At the beginning of the phase two, the team consisted of two developers, a UX specialist and a project manager. After a one third of the period, a third developer was included in the team. These two phases took approximately 56\% of the total budget. According to the Pivotal Tracker, 55 new features were implemented and 2 bugs were fixed. New source code was worth of 285 function points, totaling 415 function points for the project so far.

% Things affecting quality

During these phases, the quality aspects of the project shifted in different directions. A staging server environment identical to the production server was created. The staging server was integrated with the continuous integration server. The staging server had two purposes: the customer could demonstrate the application to potential customers and some system testing could be done in the staging server. 

% * Vaihe 4
% 	* Aikataulu: 5.6.2013 - 15.7.2013
% 	* Budjetti: 6%   15 000€
%	* Function points 				475 (+50)
% 	* Pääkontentti: 
% 		* Perhepäivähoidon toiminnot ja rajapinnat 
% 	* Tiimi
% 		* Osmo
% 		* Mike (osa-aikainen)
% 		* Miro
% 		* Kimmo
% 		* (Elice)
\textbf{Phase four.}

% Content
% Time frame, team, budget, new features and bugs, function points
% Things affecting quality


% * Vaihe 5
% 	* Aikataulu: 15.7.2013 - 16.7.2013
% 	* Budjetti: 12%   30 000€
%	* Function points 				500 (+25)
% 	* Pääkontentti: 
% 		* Läsnäolotietomallin uudistus, läsnäolotietojen esitys ja muokkaus, kulukorvaukset kunta UI:hin 
% 	* Tiimi
% 		* Osmo (osa-aikainen)
% 		* Mike
% 		* Miro
% 		* Kimmo
% 		* (Elice)
\textbf{Phase five.}

% Content
% Time frame, team, budget, new features and bugs, function points
% Things affecting quality

% http://www.qsm.com/resources/function-point-languages-table
% Function points







% kokeilut, testausta siirrettiin tulevaisuuteen
% featuret edellä, testaus jälkikäteen
% asioiden validointi, epäselvät speksit -> tuntui turhalta testata ilman tietoa onko toiminnallisuus järkevää

