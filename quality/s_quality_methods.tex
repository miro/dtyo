
 \section{Methods for improving quality}
 
Quality of a software product can be influenced throughout its life cycle with multiple approaches. Pursuing high quality means having methods in use for both decreasing the amount of defects and maintaining good structural quality. In addition to these technical approaches, projects should have methods to assure high quality of specification and implementation process.

Methods for improving quality can be split into defect prevention methods, pretest methods, testing and post-release methods. Preventive and pretest methods contain actions such as reviews, inspections and audits. Testing includes various types of testing the functionality. Post-release methods focus on maintainability, defect discovery and defect repairing.

A solid base for high quality is built with good specification, requirements and planning in the beginning of a project. When development begins, there should be ways to prevent as much defects as possible. As defects appear anyway, they should be detected and fixed as early as possible. Detecting the defects early lowers the effort needed to fix them. 

The hardest defects to remove are found in the requirements and design because testing and static analysis cannot find them. This is because these defects tend to be deficiency of features and errors of logic rather than errors in code. These defects are usually situated in the beginning of the life-cycle and thus require great effort to be removed. 

At some point, the project can initiate testing phase. Testing is used to systematically find defects in the software. Tests can be aimed to different areas of the software and the range of different tests used is dependent of the project. Big and critical projects should use comprehensive testing whereas smaller projects can get along with smaller amount of tests and lesser coverage.

Quality cannot be forgotten when the software is released. Because defect removal efficiency can never reach 100\%, there are always defects after the release. Some of those defects may have been found in the previous phases, but not removed, and other defects were unknown to the project team at the time of the release. Quality methods after the release should include detecting and removing defects and increasing the design for increasing the maintainability. With this range of methods divided to the software life-cycle, every project should choose the most appropriate methods to be used in the project in question.



ROI: Three main activities: Review, process audit and testing



MITÄ TEHDÄÄN


 \subsection{Preventing defects}

Defect prevention is a set of methods used to lower the amount of defects coming from one or many of the defect origins. Software defects are originated from different parts of the project. Defect origins can be technical and nontechnical. Some nontechnical origins of defects include requirements and documentation. Technical origins include architecture, design and code. Because defects are originated from multiple sources, there is no single method for covering them all. Most methods are not effective against all sources. Usually there are 1-4 methods used.

Most of the defect prevention methods are not primarily used for preventing defects, but for some other purpose. Preventing the defects is usually a secondary effect and can be sometimes incidental. These methods don't affect structural quality, but the defects of the software. For achieving high total quality of software, defect prevention should be combined with other types of quality methods.

Defect prevention is one of the most difficult topics of software quality. It is hard to measure, improve and prove the economic value. The difficulty originates from the fact that defect prevention is a negative factor that reduces defect potentials. This means that reliable measurement of the efficiency needs multiple points of reference for both using the method and not using it. Despite this, a number of big companies has studied the topic with significant amounts of projects. IBM, for one example, has been studying this topic since its first studies in 1970s. 

% Inspection

\textbf{Formal inspections} were inspected as a one line of research by IBM in the 1970s. The inspections were targeted at requirements, design documents, source code and other deliverables. In a short time they discovered that the defect removal efficiency with formal inspections could reach levels beyond 85\%. At that time that was higher than any form of testing could achieve. In addition, the inspections seemed to affect the accuracy and completeness of the requirements and specification documents, which lead to raising the defect removal efficiency of testing by 5\%. Combining formal inspections with formal testing could raise the efficiency even further to levels as high as 97\%. 

With these improvements in efficiency, the amount of defects in the beginning of testing was reduced significantly. The schedules and budgets of the testing could be cut to half and in some cases even more than half. That lead to about 15\% decrease in combined schedule and cumulative effort compared to similar applications without inspections.

Another result from the studies was that using the formal inspections for a longer time, the project teams unconsciously started avoiding the kind of errors found in the inspections. This meant that the inspections not only removed defects but also prevented them from occurring.

% TODO: Jotain nykypäivästä?

% Agile development method p.136 / Agile manifesto?
\textbf{Agile development method} is a set of guidelines based on a publication made by 17 software developers. The developers had gathered to discuss about lightweight development methods and the result was the published in The Agile Manifesto~\cite{beck2001agile}. The people in the signature of the manifest formed the Agile Software Development Alliance.

The Agile Manifesto reads as follows:

\begin{quote}

	"Seventeen anarchists agree: 

	We are uncovering better ways of developing software by doing it and helping others do it. Through this work we have come to value: 

	\begin{itemize}
	\item Individuals and interactions over processes and tools.
	\item Working software over comprehensive documentation.
	\item Customer collaboration over contract negotiation.
	\item Responding to change over following a plan.
	\end{itemize}

	That is, while we value the items on the right, we value the items on the left more.

	We follow the following principles:
	\begin{itemize}
	\item Our highest priority is to satisfy the customer through early and continuous delivery of valuable software. 

	\item Welcome changing requirements, even late in development. Agile processes harness change for the customer's competitive advantage. 

	\item Deliver working software frequently, from a couple of weeks to a couple of months, with a preference to the shorter timescale. 

	\item Business people and developers work together daily throughout the project.  
	\item Build projects around motivated individuals. Give them the environment and support they need, and trust them to get the job done. 

	\item The most efficient and effective method of conveying information to and within a development team is face-to-face conversation. 

	\item Working software is the primary measure of progress.  
	\item Agile processes promote sustainable development. The sponsors, developers and users should be able to maintain a constant pace indefinitely. 

	\item Continuous attention to technical excellence and good design enhances agility.  
	\item Simplicity—the art of maximizing the amount of work not done—is essential.  
	\item The best architectures, requirements and designs emerge from self-organizing teams.  
	\item At regular intervals, the team reflects on how to become more effective, then tunes and adjusts its behavior accordingly."

	\end{itemize}

\end{quote}

These methods are based on the actual methods present in the work done by the 17 developers in that time. The purpose of the manifest is not to give definite answers but some guidelines of how to prefer the aspects of software development. It is not trying tell how things are done but help developers with agile methods. When listing the things to value, the purpose is not to underrate the aspects with lower priority but give some hints about what brings the most value to the project.

% TODO: Principlet avattuna paremmin ??????



% Test driven development TDD book?
\textbf{Test-driven development} (TDD) is a development process where the development consists of short repetitious cycles of development. A cycle comprises an initially failing test case, minimum amount of code to pass that test and refactoring of the code. The development process embraces the phrase "clean code that works" by Ron Jeffries. Kent Beck analyses the benefits of that statement in his book "Test-driven Development: By Example".

Writing clean code that works can help developers by allowing a predictable flow of development. Using tests to define the finished state of a task helps developers know when the task is finished. This is contrary to a common way of development, where developers may be uncertain whether the task is finished or is there still some trail of bugs to fix. Another benefit is that when the developers aims to clean code, instead of building the first thing they think of, they can learn different sides of the problem thinking about another solutions. These benefits lead to enhancing the lives of the users and developers and the whole team. The project team can achieve better trust between the developers and the individual developers can feel better when writing clean code.

Writing clean code that works is not such an easy task. Anyone working in the software development can admit that there are many forces driving the development further from clean code and even from code that works. One solution is using automated tests as the driving force of the development. This is called Test-Driven Development. There are two cornerstones in TDD: new code is written only if an automated test has failed and duplication is eliminated. These rules seem simple enough, but they can actually produce complex behavior for individuals and the whole team. The team must be able to choose between decisions by getting feedback from the running code. Every developer must write his own tests in opposite to waiting for someone else to write the tests. The development environment must be quick enough to provide instant feedback on small changes. The design of the software must allow easy testing by using simple, loosely coupled components.

These complex requirements imply a specific order of activities in development. TDD defines the steps of development and the order of executing them as Red, Green and Refactor. Red and green are the colors of the test success. First a simple test is written so that it won't succeed. The test won't sometimes even compile. Then the test is made green, successful, by not avoiding any means necessary. After the green is achieved, the code is refactored to eliminate all of the duplication created while still keeping the test green.~\cite{beck2003test}






	% Clean code: Incrementalism TDD p. 213




% Automated risk analysis (p. 177-184)
% TODO: \textbf{Automated risk analysis} 



% Embedded users p.136 p.58

\textbf{Embedded users} is an Agile method where one or more user representatives are embedded into the project team. The purpose of the user representative is to work in cooperation with the developers creating the critical requirements which are then implemented in short sprints. The idea is to build the specific business critical features and get those running as quickly as possible. The embedded customer representatives are also used to give support in reviewing the features and requirements. The main purpose of this method is to improve the requirements definition.

This method is proven to be useful in small software projects under 2500 function points and effective in projects with under 100 users and size below 1000 function points. In larger scale applications, over 10000 function points or more than 1000 user, a single representative cannot effectively provide enough requirements. This method can however be scaled up by using multiple user representatives, but like other Agile methods, this is most effective in smaller projects.


% Static analysis p. 185 (Automated p. 267-276)

\textbf{Static analysis} is used to detecting syntactic and structural defects in source code without compiling or executing it. It is used in all sizes of software projects and with every type of applications. The concept is originally from the compilers, which performed syntax checking. Later in the 1970s, the features for detecting defects were improved in a tool called Lint. Static analysis has been further developed over time and nowadays it can comprehensively analyze system-level structure and even security vulnerabilities. Modern tools for static analysis can have defect removal efficiency of over 85\%.

Static analysis tools are based on a library of rules defining the conditions to be examined. Some of the modern commercial tools contain over 1500 rules and allow the users to define their own rules for special conditions. Static analysis is effective in defect removal for using the rules to seek out and eliminate syntactic and structural defects. There are two reasons static analysis is also useful in preventing defects: the rule libraries are also useful for preventing defects and the tools can suggest corrections for defects to the developers. The latter enables the developers to see the effective solutions to the defects while examining the results of the analysis.

Automated static analysis of source code is used in both defect prevention and pretest defect removal. Statistics by Capers Jones show that the usage of automated static analysis exceeds 75\% in most types of software projects. The first mentions of automated static analysis are from 1979 from the first release of Lint. In current modern software development, static source code analysis is automatically done by most of the Integrated Development Environments (IDE). IDEs, such as Eclipse and IntelliJ Idea, perform automatic analysis immediately after every minor change.

Static analysis tools are not only very quick and effective but also fairly inexpensive. Because of this, static code analysis has become one of the most used quality methods in software industry. There are tens or even more tools for source code static analysis in the market, both open source and commercial. For such an inexpensive and effective method, one could imagine the market penetration being close to 100\%. Jones suggests that the reason for this not being true is that humans have a natural resistance for new ideas even though the turn out to be valuable.

 \subsection{Pretest methods}

Capers Jones suggests that in every software project there should be multiple pretest QA methods used. Jones lists a combination of methods for both small and large projects. A small software project, in this context, is described to have a maximum amount of 1000 function points or 50 000 source code statements. These small projects are generally executed by a team with less than 6 software developers. These teams usually have no specialists for any quality methods, but the developers are generalists handling requirements, design, coding and testing. In many cases with Agile approach, there are users representative embedded in the team providing requirements and customers viewpoint in real time. Jones reminds that removing defects with high efficiency requires trained and technically skilled software engineers instead of generalists. However, this is not as necessary in small projects, since fortunately these projects have usually low defect potentials.

The origins of defects in small studied projects are split into five categories. Source code is the most common origin of defects. About 1.75 defects per function point are found in source code and this leads to 1750 defects in whole projects. Software design is the second most common source of defects. Design is the origin of 1.00 defects per function point. Requirements are causing 0.75 defects per function point and documentation nearly as much with 0.65 defects. Poorly executed fixes are the origin of 0.27 defects per function point. All together these five are the source for 4420 defects in a whole 1000 function point software project. These figures represent the approximate averages and the actual values can be as much as 25\% lower or higher for every source. 

Jones presents a suite of pretest defect removal activities and their efficiencies for small projects. This suite consists of five methods:

\begin{enumerate}

\item Personal desk checking

\item Scrum sessions

\item Client reviews of specifications

\item Informal peer reviews

\item Static analysis of source code

\end{enumerate}

Each of these forms of defect removal activities are targeted towards a specific type of defects, but other types of defects can be found during the activities. Jones gives several figures for the efficiency of each activity. These figures can only be created by companies that have complete accurate defect measurement programs. Because of this, these figures can vary from context to other and thus are indicative. These figures still illustrate two major problems in the software industry: the removal efficiency levels are comparatively low for most of the removal activities and the defect removal is much harder for requirements and design. The first one leads to a need for numerous kinds of defect removal activities. The latter means that a significant amount of effort must be used to assure the quality of requirements and design. Defects in requirements and design must be removed prior to testing, because the testing cannot find them. Also static analysis is incapable to finding them, because the defects are not bugs found in the code.


% STATSIT: p.198

% Desk checking p. 208
\textbf{Personal desk checking} is a manual operation in where the logic of an algorithm is checked by the creator of the algorithm . The logic used in the desk checking is presented as a pseudo-code rather than the implemented actual program code. The algorithm is executed by a person acting as a computer. The person performing the desk checking carefully follows the algorithm while filling a table of notes with pen and paper. 

The notes form a table which include columns for: line number, variables in use, conditions, input and output. Line number is necessary to identify the line being executed. All variables have a column in alphabetical order. As the value of the variable changes, the appropriate column is filled up. Conditions columns include a column for every condition in the algorithm which shows the result of the condition in either true(T) or false (F). The condition column is updated whenever the condition is evaluated. Input and output columns are used for the inputs got from the user and the output from the program.~\cite{campionDescCheck}

Desk checking is the oldest form of software defect removal. It has been in use since the beginning of the history of computers. In the early days of computing, testing the programs was difficult because of the limited numbers of computers. The computers worked on production work in the daytime and often the testing had to be done at night. In those days, testing had only an efficiency of 70\% in finding bugs, because of the primitive test case design and limited time available for testing. Therefore the desk checking were a necessary addition to testing. Nowadays the desk checking is still a common activity for removing defects prior to testing. Desk checking today can be enhanced by using static analysis for program code and spell checkers and complexity tools for text documents.

% Figures:
Personal desk checking is used mainly in low-level code. Approximately over 75\% of low-level code and under 30\% of high-level code in projects are checked using personal desk checking. Additionally, personal desk checking is used for over 75\% of text documents, such as requirements. The execution time of desk checking is around 80\% of normal reading speed of text. This leads to about 5 logical statements per minute for source code.

The efficiency of defect removal for personal desk checking is between 25\% and 50\% averaging 35\%. The reason for these relatively low figures can be found in human nature. Humans have a natural tendency to ignore their own mistakes. A developer making an error usually does the error thinking that the action was correct. Therefore when the developer checks the code for defects, the train of thought can remain the same and the defect is not found. This could be avoided by using proofreaders or copy editors, which is a rare habit, but could be profitable for software projects. Another solution for avoiding the blindness to own mistakes is to use peer reviews described later.

% Scrum sessions p. 222
\textbf{Scrum sessions }

% Scrum <- rugby
% Teams of generalists:
% scrum-master, around 5 developers, embedded user/stakeholder
% 2 week sprints: embedded user provides the requirements
% after each sprint.. source code and supporting documentation should be ready to go production
% Pretest: daily scrum meetups/stand-ups, 15 minutes:
%  	-whats done
%	-what next
%	-problems, bugs, issues slowing them down <- relevant to pretest defect removal
% TDD
% For up to 2000 fp -> popular and succesful

% effectivity can be better than in figures, because capturing the defect removal data is not present in agile

% p. 235
\textbf{Client reviews}

% Peer review p. 209
\textbf{Peer reviews}

% Static analysis p. 267
\textbf{Static analysis}






-ROI: Three main activities: Review, process audit and testing

 

 \subsection{Testing}
 
 \begin{itemize}
 
 \item Crash, Smoke and Kattava testaus
 
 \item ECO: Chapter 5
 
 \end{itemize}
 
\subsection{Post release}
 

