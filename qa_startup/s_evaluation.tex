
 \section{Evaluation of QA Methods}

Quality assurance in a startup environment differs from the QA activities in traditional software development. This is because the requirements and goals of a startup project are usually considerably different from a traditional software projects. Furthermore, even the term software quality can be partially redefined for a better fit in a software startup. Eric Ries describes that the development of Minimum Viable Products questions the traditional notions of quality and this is the most vexing aspects of the development. TODO: Viite

% Cost of owning a mess Clean Code p. 4

 \subsection{Quality in a Startup project}

% Lean Startup: p. 107-110
In the beginning of the chapter about role of quality, Eric Ries states that "The best professionals and craftpersons alike aspire to build quality products; it is a point of pride". This is a good summary about the attitude towards software quality in many movements about modern software development. Measuring and defining quality in modern software projects with constant changes and uncertainty can be difficult, but the development personnel should have the pursuit for high quality.

Ries tells that modern production processes seek to boost efficiency by relying to high quality. The belief that the customer is the most important part of the production process means that all effort should be focused to producing results that the customer finds valuable. This view can be beneficial in an environment where the company knows the opinions of customers. However, in a startup development, assuming the opinions of customers is a risky thing to do. Often in the startup, it is not even sure who the customer is. Thus, Ries introduces a quality principle for startups: "If we do not know who the customer is, we do not know what quality is".

In a startup developing an MVP, the quality of the product can be a fluid concept. Even if the quality of MVP is low for customers, it can bring great value in building a high-quality product. If the customers find the product low on quality, this can be used as an opportunity to learn what customers care about. This is infinitely better than mere speculation, because it provides empirical information on which to build future products.

When working with a 3D chat software called IMVU, Ries and his colleagues decided to leave a critical sophisticated feature done with only minimum effort. They were embarrassed to release the moving of the avatars without any animations or other modern visualizations. The avatars just reappeared to another location. The response of the customers was surprising as the customers were thrilled from the new feature, which allowed an immediate change of location without waiting. From the customers point of view, the released feature was more appropriate than the option which would take more time and money to implement. In the end, the quality of the released feature was probably higher than the one planned. The lesson behind the story is that customers do not care how much time something takes to build.

Lean Startup method is aiming for the goal of winning over customers and not opposed to building high-quality products. Thus, it is necessary to set aside some professional standards to enable the validated learning as soon as possible. This is not supposed to allow operating in an undisciplined manner. This is important as there are some quality problems that can slow down the Build-Measure-Learn feedback loop. In addition, defects complicate the evolution of the product and interfere with the ability to learn. Helping the development of the MVP means removing any feature, process or effort that does not lead directly to the learning sought.

Another view of the software quality applicable to a startup environment is described in the Lean Software Development method.~\cite{poppendieck2003LSD}

% LSD: p. 16-

% Laatu painottuu startupissa siihen, mitä asiakkaat ja loppukäyttäjät pitävät tärkeänä
% Laadun merkitys elinkaaren eri vaiheissa muuttuu
%	-Aluksi tärkeintä validoida featuret ja idea
%	-Laadussa tärkeää aluksi hyvä muokattavuus, refaktoroinnit, asiakkaalle kriittisten osien testaus
%	-Elinkaaren alussa ei väliä matalan prioriteetin bugeilla tai featureiden vajaalla toiminnalla
%	-Kun idea alkaa olla validoitu, muuttuu laadun merkitys tuotteistuksen yhteydessä
%	-Laatuun on käytössä myöhemmin enemmän resursseja, koska asiakkaita alkaa olla
%	-Osuiskohan tähän Lean Startupin skaalausvaihe






% Quality p.106-110
% MVP voi olla "low-quality", jos rakennetaan "high-quality" tuotetta
% Asiakkaan mielestä MVP voi olla low quality, mutta siitä voidaan oppia mistä ominaisuuksista asiakas pitää
% "If we dont know who the customer is, we do not know what quality is"
% 














%%%%%%%%%%%%%%%%%%%%%%%%%%%%%%%%%%%%%%%%%%%%%%%%%%%%%%%%%%%%%%%%%%%%%%%%%%%%%%%

MILLÄ MENESTYKSELLÄ TEHDÄÄN JA KUINKA SOPII TÄHÄN SCOPEEN



PREVENTIVE
-Small business applications p.126
	-1000fp
	-embedded users
	-agile development method
	-TDD
	-automated risk analysis
	-static analysis on all code segments
	-This combination should lower defect potentials by 45\% and ensure defect removal 95\%




 \begin{itemize}
  
 \item ECO: We use these quality metrics to compare a number of quality improvement techniques at each stage of the software development life cycle and quantify their efficacy using data from real-world applications.
 
 \item Direct costs \& efforts p. 200 table

 \item ECO: Analysis of pretest defect removal activities p. 208

 \end{itemize}
 
 \subsection{Building Quality In}

%http://books.google.fi/books?id=RTt9AgAAQBAJ&lpg=PA27&ots=PXXKjr2e_E&dq=%22According%20to%20Shigeo%20Shingo%2C%20there%20are%20two%20kinds%20of%20inspection%22&pg=PA29#v=onepage&q&f=false
% In Lean Software Development, the goal is to build quality into the code from the start, not test it in later. Dont focus on putting defects into a tracking system, you avoid creating defects in the first place. It takes a higly disciplined organization to do that. p26
% There are two kinds of inspections, inspection after defects occur and inspection to prevent defects. If you really want quality, you don't inspect after the fact, you control conditions so as not to allow defects in the first place. If this is not possible, the you inspect the product after each small stemp, so that defects are caught immediately after they occur. When a defect is found, you stop the line, find its cause and fix it immediately. p.27
% Defect tracking systems are queues of partially done work, queues of rework. In Lean, queues are collection points for waste.



{poppendieck2006implementing} 

 \subsection{Pretest defect removal}
 
 \subsection{Testing}
 
 \subsection{The Don'ts - Things To Avoid}
 
 \begin{itemize}
 
 \item Liittyy vahvasti Leanin Wasteen
 \item Älä raportoi bugeja, joista tiedät, ettei niitä korjata
 \item Ei turhia raportteja
 \item ECO: Cost per defect => paras tulos bugisimmassa projektissa
 
 \end{itemize}

% ECO:  p. 127 harmful combinations PREVENTIVE

Capers Jones has interpreted from the research of multiple years of software quality that there can be harmful combinations of methods used. Even when combinind harmful methods with helpful ones, the harmful method seem to end up winning. That is, defect potentials are raising instead of coming down. Some methods combined with others can raise the defect potentials and make applications risky with a change of failure.
